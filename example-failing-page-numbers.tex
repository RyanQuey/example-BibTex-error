% !TeX program = lualatex 
\documentclass[A4,12pt]{article}

\usepackage[bidi=basic]{babel}
\babelprovide[import=en-US,main]{american}
\babelprovide[import=he]{hebrew}
\babelprovide[import=el]{polutonikogreek}
\babelfont[american]{rm}[Ligatures=TeX]{Linux Libertine O}
\babelfont[hebrew]{rm}%
  [Ligatures=TeX,Contextuals=Alternate]{SBL BibLit}
\babelfont[polutonikogreek]{rm}%
  [Ligatures=TeX,Contextuals=Alternate]{SBL BibLit}
\usepackage{csquotes}
\PassOptionsToPackage{indexing=cite}{biblatex}
% Pass jblstyle to sbl-paper if you want double spaced footnotes
\usepackage{sbl-paper}

\addbibresource{./bibliography/example.bib}
\nocite{*}

\makeatletter


 \renewcommand\@makefntext[1]{%
     \parindent\footnotemargin%
     \@thefnmark.\@\space
     #1
  }

\makeatother  



% ---------------------------------------------------------------------
% Index set up (optional)
% Include separate indices for subjects, authors, and Scripture
% references

% Set up subject index
\makeindex[title=Subject Index,intoc,options=-q]

% Set up Scripture reference index
\makeindex[name=\jobname-scr,title=Scripture Reference Index,intoc,
           options=-s sbl-paper-bibleref.ist -q]
% Index Bible references in a scripture index
\renewcommand{\biblerefindex}{\index[\jobname-scr]}

% Set up Author index
\makeindex[name=\jobname-aut,title=Author Index,intoc,options=-q]
% Redefine index format so biblatex inserts names into a separate
% author index rather than the main index
\DeclareIndexNameFormat{default}{%
  \nameparts{#1}%
  \usebibmacro{index:name}
    {\index[\jobname-aut]}
    {\namepartfamily}
    {\namepartgiven}
    {\namepartprefix}
    {\namepartsuffix}}
% ---------------------------------------------------------------------

%% These might be good, but trying to just copy bibtex-sbl first
%\usepackage{xcolor}

% https://tex.stackexchange.com/a/17004/327861
\usepackage{pdfcomment}



\hypersetup{%
  pdftitle={Luke's Use of Leviticus' and Isaiah's Theology of Atonement and Sabbath in Luke 4:18--19},
  pdfsubject={Christ as the Spirit-Anointed Proclaimer of the Jubilee},
  pdfauthor={Ryan Quey},
  pdfkeywords={Luke 4, Isaiah 53, Isaiah 61, Isaiah 58, Leviticus 16, Leviticus 23-27, Sabbath, Year of Jubilee, Multi-Layered, Intertextuality}}



\begin{document}



\hypersetup{pageanchor=false}%
%\newgeometry{margin=2in}%
\newgeometry{margin=1in}%
\thispagestyle{empty}%
\singlespacing
\begin{center}

  \vfill

  \doublespacing
  \MakeTextUppercase{
    Example Title
  }

  \vfill
  
  \singlespacing
  \MakeTextUppercase{
    A Research Proposal\\
     for the Degree of\\
     Doctor of Philosophy
  }

  \vfill

  \MakeTextUppercase{
    By Example Author\\
    December 23, 2024
  }
  
\end{center}
\restoregeometry
\hypersetup{pageanchor=true}%




\pagenumbering{roman}

\clearpage
\pagenumbering{arabic}

\firstsection[Proposal Heading]{example heading}

\index{margins}%


\subsection{Subsection}

  
Example text.\footnote{
  Some working pagination: \cites[497]{oswaltBookIsaiahChapters1998}[268-269]{kimReadingIsaiahLiterary2016}. 
  A breaking pagination: \cite[51-55]{bakerJubileeMillenniumHoly1998}. % TODO not showing hte page number for some reason
}

But what if I put in same entry again?\footnote{
  A no longer breaking pagination: \cite[51-55]{bakerJubileeMillenniumHoly1998}. % TODO not showing hte page number for some reason
} 


Another breaking example I found:\footnote{
  Why does the first one break? \cite[312]{akagiAcceptabilityPurityActs2024}.
}
Again, it works on second try\footnote{
  But the second doesn't? \cite[312]{akagiAcceptabilityPurityActs2024}.
}
%\hypertarget{justification-for-research}

This topic thesis has methodological, exegetical, and theological significance. Methodologically speaking, the present study is significant for its application of recent advancements in hermeneutics and semiotics to a particularly complex, multilayered case of intertextuality in the NT.\footnote
  {
    See pg. 1--2n4--6 above.
  }

The application of this methodology will also be shown to have exegetical significance for the Gospel of Luke. Much has been written on the relationship between Luke's use of Isaiah\footnote
  {
    E.g., \cites{beersFollowersJesusServant2015}{paoLuke2007}{sandersIsaiah61Luke1993}{koetIsaiahLukeActs2005}{smitFunctionTwoQuotations2013}{mullerReceptionOldTestament2001}.
  }
both in regard to Luke 4:18--19 and in Luke-Acts as a whole,\footnote
  {
    See e.g. \cite[306-315]{witheringtonIsaiahOldNew2017}. ``Luke of course is the only Gospel writer who provides us with a sequel, and it is clear from his use of Isaiah in Acts . . . that he sees that prophetic book as just as important to the telling of the story of earliest Christianity as to the telling of the story of Jesus.  Furthermore, Isaiah provides a clue of just how carefully crafted these two volumes are, and how they are intended to be read together, for there are three clear instances where Luke will simply allude to an Isaiah text in the Gospel and then cite it in Acts. . . . Even more revealing of Luke's perspective on things is the fact that in the speeches and dialogues in Acts Isaiah is explicitly quoted only when Jews are the audience. Clearly Luke is both careful and skillful in how he uses the prophetic oracles of Isaiah. He uses Isaiah to some degree as a framework for his own two volumes.''
  } 
as well as parallels that this has with writings from 2\textsuperscript{nd} Temple Judaism (such as 11Q13 and 4Q521).\footnote
  {
    \cites[310]{witheringtonIsaiahOldNew2017}[213]{greenGospelLuke1997}.
  }
However, relatively little work has been done to draw these lines back further to the theology of Sabbath in Leviticus.\footnote
  {
    Perhaps most conspicuous exception is Sloan's work on Isa 61. See \cite{sloanjr.FavorableYearLord1977}. However, Sloan's focus is now outdated, and also does not interact with hermeneutical issues or the broader contexts of Isa 61 and Lev 23--27, including the theology of Sabbath as much.
  }

Specifically, although commentaries on Luke note the connection to the Year of Jubilee in Leviticus 25, there is relatively little exploration of how the Year of Jubilee relates to the theology of Sabbath and thereby to atonement within the internal logic of Leviticus. The particular relationship between these concepts as developed in Leviticus and Isaiah provide explanatory value for how the forgiveness of sins relates to personal and cosmic salvation in Luke (e.g., Luke 1:76), as well as Sabbath-related passages in Luke (cf. 6:1--11; 13:10--17; 14:1--6; 23:54--24:1).\footnote
  {
    This study will not be the first to notice connections between the Sabbath day and New Creation theology in Luke, but will seek to develop it further. See for example Green's comments on 4:40--41; 6:8--10; 19:40; 23:43 in Green, \emph{Luke}, 226, 255, 688, 823. Green connects Sabbath to New Creation (``What, then, is the nexus between Sabbath and healing in this instance?  Jesus' ministry `. . . restores to the sabbath command its profound significance: \emph{restoration} of human beings in their integrity as part of God's creation'\,'') and even notes how Luke 6:1--11 is part of Luke's expansion on the nature of Jesus' ministry as the ``onset of the eschatological Jubilee'' from Luke 4:18--19. However, he fails to adequately draw out the significance of the Year of Jubilee's connection to Sabbath against the backdrop of Lev 25.
  }

This study will also provide greater clarity as to the rationale behind Jesus' incorporation of Isaiah 58:6 in Luke 4:18 by showing how Isaiah 58:6 and 61:1--3 are linked together by the theological relationship between atonement, Sabbath and the year of Jubilee in Leviticus.\footnote
  {
    For example, note how Isa 58 alludes to the Sabbath (58:13--14), Day of Atonement (58:1--7), and the cosmic restoration of the land (58:11--12), a point that is missed by many Luke commentaries.
  } 
Furthermore, this study will also have implications beyond just Luke 4:18--19 due to the programmatic role that 4:18--19 has in regard to Luke's portrayal of Jesus' ministry overall. Accordingly, this study will also be theologically significant and will contribute to our understanding of the atonement's relationship to a biblical theology of Sabbath and Jubilee relative to Jesus' ministry.

\printbibliography[heading=bibintoc]

%\singlespacing

%\printindex

%\printindex[\jobname-scr]

%\printindex[\jobname-aut]

\end{document}
