% !TeX program = lualatex 
\documentclass[A4,12pt]{article}

\usepackage[bidi=basic]{babel}
\babelprovide[import=en-US,main]{american}
\babelprovide[import=he]{hebrew}
\babelprovide[import=el]{polutonikogreek}
\babelfont[american]{rm}[Ligatures=TeX]{Linux Libertine O}
\babelfont[hebrew]{rm}%
  [Ligatures=TeX,Contextuals=Alternate]{SBL BibLit}
\babelfont[polutonikogreek]{rm}%
  [Ligatures=TeX,Contextuals=Alternate]{SBL BibLit}
\usepackage{csquotes}
\PassOptionsToPackage{indexing=cite}{biblatex}
% Pass jblstyle to sbl-paper if you want double spaced footnotes
\usepackage{sbl-paper}

\addbibresource{./bibliography/example.bib}
\nocite{*}

\makeatletter


 \renewcommand\@makefntext[1]{%
     \parindent\footnotemargin%
     \@thefnmark.\@\space
     #1
  }

\makeatother  



% ---------------------------------------------------------------------
% Index set up (optional)
% Include separate indices for subjects, authors, and Scripture
% references

% Set up subject index
\makeindex[title=Subject Index,intoc,options=-q]

% Set up Scripture reference index
\makeindex[name=\jobname-scr,title=Scripture Reference Index,intoc,
           options=-s sbl-paper-bibleref.ist -q]
% Index Bible references in a scripture index
\renewcommand{\biblerefindex}{\index[\jobname-scr]}

% Set up Author index
\makeindex[name=\jobname-aut,title=Author Index,intoc,options=-q]
% Redefine index format so biblatex inserts names into a separate
% author index rather than the main index
\DeclareIndexNameFormat{default}{%
  \nameparts{#1}%
  \usebibmacro{index:name}
    {\index[\jobname-aut]}
    {\namepartfamily}
    {\namepartgiven}
    {\namepartprefix}
    {\namepartsuffix}}
% ---------------------------------------------------------------------

%% These might be good, but trying to just copy bibtex-sbl first
%\usepackage{xcolor}

% https://tex.stackexchange.com/a/17004/327861
\usepackage{pdfcomment}



\hypersetup{%
  pdftitle={Luke's Use of Leviticus' and Isaiah's Theology of Atonement and Sabbath in Luke 4:18--19},
  pdfsubject={Christ as the Spirit-Anointed Proclaimer of the Jubilee},
  pdfauthor={Ryan Quey},
  pdfkeywords={Luke 4, Isaiah 53, Isaiah 61, Isaiah 58, Leviticus 16, Leviticus 23-27, Sabbath, Year of Jubilee, Multi-Layered, Intertextuality}}



\begin{document}



\hypersetup{pageanchor=false}%
%\newgeometry{margin=2in}%
\newgeometry{margin=1in}%
\thispagestyle{empty}%
\singlespacing
\begin{center}

  \vfill

  \doublespacing
  \MakeTextUppercase{
    Example Title
  }

  \vfill
  
  \singlespacing
  \MakeTextUppercase{
    A Research Proposal\\
     for the Degree of\\
     Doctor of Philosophy
  }

  \vfill

  \MakeTextUppercase{
    By Example Author\\
    December 23, 2024
  }
  
\end{center}
\restoregeometry
\hypersetup{pageanchor=true}%




\pagenumbering{roman}

\clearpage
\pagenumbering{arabic}

\firstsection[Proposal Heading]{example heading}

\index{margins}%


\subsection{Subsection}

  
Example text.\footnote{
  Some working pagination: \cites[497]{oswaltBookIsaiahChapters1998}[268-269]{kimReadingIsaiahLiterary2016}. 
  A breaking pagination: \cite[51-55]{bakerJubileeMillenniumHoly1998}. % TODO not showing hte page number for some reason
}

But what if I put in same entry again?\footnote{
  A no longer breaking pagination: \cite[51-55]{bakerJubileeMillenniumHoly1998}. % TODO not showing hte page number for some reason
} 


Another breaking example I found:\footnote{
  Why does the first one break? \cite[312]{akagiAcceptabilityPurityActs2024}.
}
Again, it works on second try\footnote{
  But the second doesn't? \cite[312]{akagiAcceptabilityPurityActs2024}.
}

\printbibliography[heading=bibintoc]

%\singlespacing

%\printindex

%\printindex[\jobname-scr]

%\printindex[\jobname-aut]

\end{document}
